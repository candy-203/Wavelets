\documentclass[]{scrartcl}

\usepackage{amsthm}
\usepackage{amssymb} %for mathbb
\usepackage{mathtools} %for coloneqq

\usepackage{enumitem}   
\usepackage{pgfplots}
\pgfplotsset{compat = newest}

\usepackage[ngerman]{babel}
\usepackage[T1]{fontenc}
\usepackage[utf8]{inputenc}

\usepackage{hyperref}

\newtheorem{theo}{Theorem}[section]

\theoremstyle{plain}
\newtheorem{satz}[theo]{Satz}
\newtheorem{kor}[theo]{Korollar}
\newtheorem{lem}[theo]{Lemma}
\newtheorem{prop}[theo]{Proposition}
\newtheorem{defn}[theo]{Definition}

\theoremstyle{remark}
\newtheorem*{bsp}{Beispiel}
\newtheorem*{bem}{Bemerkung}

\newcommand{\satzautorefname}{Satz}
\newcommand{\propautorefname}{Proposition}
\newcommand{\lemautorefname}{Lemma}
\newcommand{\defnautorefname}{De\-fi\-ni\-tion}
\newcommand{\korautorefname}{Korollar}
\newcommand{\theoautorefname}{}

\newcommand{\lZ}{l^2(\mathbb{Z})}
\newcommand{\lR}{L^2(\mathbb{R})}

\title{Wavelets}
\author{Jan Kandyba}
\date{10. Mai 2022}

\begin{document}
    \maketitle
    %\tableofcontents

    \section{Haar Wavelet}

    Wir versuchen eine gegebene Funktion zu approximieren. Zuerst durch eine Skallierungsfunktion $\varphi$, dann durch ein \emph{Wavelet} $\psi$.
    Nehmen wir folgende Funktionen:
    $$\varphi(t) = 
    \begin{cases}
        1, &falls \: t \in [0, 1)\\
        0, &sonst    
    \end{cases},
    \psi(t) = 
    \begin{cases}
        1, &falls \: t \in [0, \frac{1}{2})\\
        -1, &falls \: t \in [\frac{1}{2}, 1) \\
        0, &sonst    
    \end{cases} 
    $$

    Nun können wir eine beliebige Funktion $f$ durch Verschiebungen und Skallierungen der Funktion $\phi$ darstellen. Dabei betrachen wir einen Approximationsfall $j$, indem wir die approximierende Funktion durch die Mittelwerte, konstant jeweils auf einem Abschnitt von $2^{-j+1}$.

    \begin{figure}
        \centering
        \begin{tikzpicture}[scale=0.75]
            \begin{axis}[samples=100,
                xmin = 0, xmax = 1.5,
                ymin = 0, ymax = 2.0]
                \addplot[] {x^2};
                \addplot[const plot] coordinates {(0,0.125) (0.5,0.625) (1,1.625) (2,1.625)};
            \end{axis}
        \end{tikzpicture}
        \caption{Approximation von $x^2$ durch eine Treppenfunktion}
    \end{figure}

    \section{Signale und Filter}

    \begin{defn}[Signalraum]
        $x = \{ x_n \}_{n \in \mathbb{Z}} \subset \mathbb{C}$ ist ein diskretes Signal.
        Wir verwenden die Energienorm für Signale:
        $$||x|| = (\sum_{k \in \mathbb{Z}} |x_k|)^{1/2}$$
        Im folgenden werden wir nur Signale mit endlicher Energie betrachten. Dafür definieren wir uns den Raum
        $$l^2(\mathbb{Z}) \coloneqq \{x: \mathbb{Z} \to \mathbb{C} : ||x|| < \infty\}$$
    \end{defn}

    \begin{bem}
        Wir bilden auf $\mathbb{C}$ ab, da wir eine Frequenz in der Form von $r \cdot e^{i\omega x}$, mit $r$ die Amplitude und $\omega$ die Frequenz darstellen.
    \end{bem}

    Um mit diesen Signalen zu arbeiten, definieren wir uns \emph{Filter} als Abbildungen von $l^2(\mathbb{Z})$ auf sich selber, mit der Notation $y = Hx, \: H: l^2(\mathbb{Z}) \to l^2(\mathbb{Z})$

    \begin{bsp}
        Ein Besipiel für einen linearen Filter (s. u.) ist der sogenannte \emph{Delay-Operator}, definiert durch
        $$y = D^nx \Leftrightarrow y_k = x_{k-n}$$
    \end{bsp}

    \begin{defn}
        Ein Filter $H$ ist \emph{LTI} (Linear, time invariant), falls
        \begin{enumerate}[label=(\roman*)]
            \item $H(x + y) = Hx + Hy$
            \item $H(ax) = aHx \: $
            \item $H(Dx) = D(Hx) \: $
        \end{enumerate}
    \end{defn}

    \begin{bem}
        Für einen LTI-Filter $H$ gilt $H(D^nx)=D^n(Hx)$
    \end{bem}

    Eine wichtiges Signal ist das sogenannte \emph{Impulssignal}:
    $$\delta_k \coloneqq
    \begin{cases}
        1, &falls \: k=0\\
        0, &sonst    
    \end{cases} 
    $$

    Ein beliebiges zeitdisretes Signal lässt sich durch das Impulssignal ausdrücken:
    $$x = \sum_n x_n D^n \delta$$

    Für ein LTI-Filter definieren wir uns die \emph{Impusantwort} $h \coloneqq H\delta$

    Wenden wir nun einen LTI-Filter $H$ auf $x$ an, bekommen wir folgene Umformung:

    \begin{align*}
        y = Hx \Leftrightarrow y &= H(\sum_n x_n D^n \delta) = \sum_n x_n H (D^n \delta) \\
            &= \sum_n x_n D^n h \coloneqq h * x
    \end{align*}

    Mit $*$ als Definition einer Faltung, da mit einsetzen folgendes gilt:

    $$y = h*x \Leftrightarrow y_k = \sum_n x_n h_{k-n}$$

    Damit können wir jeden LTI-Filter durch eine Folge als seine Impulsantwort definieren.

    \begin{defn}
        Die zeitdiskrete Fouriertransformation für ein zeitdirkretes Signal $x$:
        $$X(\omega) = \sum_{k=-\infty}^{\infty}x_k e^{-i\omega k}$$
    \end{defn}

    Die Fouriertransformation der Impulsantwort eines LTI-Filters heißt Frequenzantwort. 
    Für den Fall einer konstanten Frequenz als Eingabe, $x_k = e^{i \omega k}$ mit $|\omega| \le \pi$ gilt:

    \begin{align*}
        y_k &= \sum_n h_n x_{k-n} = \sum_n h_n e^{i \omega (k - n)} \\
        &= e^{i \omega k} \sum_n h_n e^{-i \omega n} = e^{i \omega k} H(\omega)
    \end{align*}

    Mit der Schreibweise $H(\omega) = |H(\omega)|e^{i \phi(\omega)}$ folgt
    $$y_k = |H(\omega)|e^{i(\omega k + \phi(\omega))}$$
    Somit ist die Ausgabe auch eine reine Frequenz, mit der Amplitude $|H(\omega)|$, und der Phasenverschiebung $-\phi(\omega)$. Die Fouriertransformation beschreibt somit durch den Betrag $H(\omega)$, inwiefern der Filter $H$ die Frequenz $\omega$ beeinflusst.

    Für den Mittelungsfilter
    $$h_k =
    \begin{cases}
        1/2, &falls \: k=0,1 \\
        0, &sonst
    \end{cases},
    y_k = \sum_n h_n x_{k-n} = \frac{x_k + x_{k-1}}{2}
    $$
    gilt $|H(\omega)| = \cos(\frac{\omega}{2}), \: |\omega| < \pi$. Wie auf \autoref{pic:Four} zu sehen, werden hohe Frequenzen mit einem Faktor von fast 0 multipliziert, wohingegen niedrige Frequenzen mit einem Faktor nahe 1 skalliert werden. So einen Filter nennen wir \emph{Tiefpaßfilter}. Bei einem gegensätzlichen Verlauf sprechen wir von einem \emph{Hochpaßfilter}.
    
    \begin{figure}
        \centering
        \begin{tikzpicture}[scale=0.75]
            \begin{axis}[samples = 100,
                xmin = 0, xmax = 3.3,
                ymin = 0, ymax = 1.0]
                \addplot[] {cos(deg(x/2))};
            \end{axis}
        \end{tikzpicture}
    \caption{Die Frequenzantwort des Mittelungsfilters \label{pic:Four}}
    \end{figure}


    \section{Multiskalenanalyse}

    Zuerst einigen wir uns auf die folgende Fouriertransformation:
    $$\hat{f}(\omega) = \int_\infty^\infty f(t) e^{-i \omega t} \, dt$$

    Unser Ziel ist es nun eine numerisch Stabile Basis des unendlich-dimensionalen Vektorraumes $L^2(\mathbb{R})$ zu konstruieren. Dazu definieren wir den Begriff der Riesz-Basis.

    \begin{defn}
        $\{\varphi_k\}_k$ eine Basis von $V \subset \lR$ eine Riesz-Basis, falls 
        $\exists A, B \in \mathbb{R}$, mit
        $$f = \sum_k c_k \varphi_k \Rightarrow A ||f||^2 \le \sum_k |c_k|^2 \le B ||f||^2$$
    \end{defn}

    \begin{bem}
        Für eine Approximation $\tilde{f} = \sum_k \tilde{c_k} \varphi_k$ für $f = \sum_k c_k \varphi_k$ gilt:
        $$A||f-\tilde{f}||^2 \le \sum_k c_k - \tilde{c_k} \le B ||f - \tilde{f}||^2$$
        Somit folgen für kleine Fehler in der Approximation auch kleine Fehler in den Koeffizienten.  
    \end{bem}

    Nun verstetigen wir den Begriff des Signals, und betrachten nun funktionen.

    \begin{defn} Der Raum, indem alle unsere Signale enthalten sind ist
        $$
        L^2(\mathbb{R}) \coloneqq \{f: \mathbb{R} \to \mathbb{C}:
        \int_{-\infty}^\infty |f|^2 \, dt < \infty\}
        $$
        mit der Norm

        $$||f|| = (\int_{-\infty}^\infty |f|^2 \, dt)^{1/2}$$

        und des daraus induzierten Skalarproduktes

        $$\langle f, g \rangle \coloneqq \int_{-\infty}^\infty f \overline{g} \, dt$$
    \end{defn}

    Nun führen wir die Multiskalenanalyse ein:

    \begin{defn}
        Eine Familie von Unterräumen $\{V_j\}_{j \in \mathbb{Z}}$ von $\lR$  heißt \emph{Multiskalenanalyse} (MSA), falls

        \begin{enumerate}[label=(\roman*)]
            \item $V_j \subset V_{j+1} \: \forall \: j \in \mathbb{Z}$ \label{MSA:i}
            \item $f(t) \in V_j \Leftrightarrow f(2t) \in V_{j+1} \: \forall \: j \in \mathbb{Z}$ \label{MSA:ii}
            \item $\bigcup_j V_j$ dicht in $L^2(\mathbb{R})$ \label{MSA:iii}
            \item $\bigcap_j V_j = \{ 0 \}$ 
            \item Es existiert eine Skalierungsfunktion $\varphi \in V_0$, mit $\{\varphi (t - k)\}_{k \in \mathbb{Z}}$ Orthonormalbasis, Riesz-Basis von $V_0$, mit 
            $\int_\mathbb{R} \varphi(t) \, dt = 1$ \label{MSA:v}
        \end{enumerate}
    \end{defn}

    Mit \autoref{MSA:ii} folgt, dass für ein $V_j$ die Familie $\{\varphi_{j,k}\}_k$ eine Basis bildet, mit $$\varphi_{j,k} (t) \coloneqq 2^{j/k} \varphi(2^j t - k)$$

    \begin{bem}
        Der Vorfaktor von $2^{j/k}$ ist notwendig, damit 
        $||\varphi_{j,k}|| = ||\varphi||$ gilt.
    \end{bem}

    Die Multiskalenanalyse ist also eine Familie von Detailräumen. Mit jedem weiteren Index werden die Signale besser approximiert. Durch \autoref{MSA:iii} bekommen wir später im Grenzübergang eine Basis von $\lR$.

    Mit \autoref{MSA:i} folgt, dass $\varphi \in V_1$. Damit ist $\varphi$ auch folgendermaßen darstellbar:
    $$\varphi(t) = 2 \sum_k h_k \varphi(2t - k)$$
    Durch Anwendung der kontinuierlichen Fouriertransformation bekommen wir mit den üblichen Rechenregeln:
    \begin{align*}
        \hat{\varphi}(\omega) &= 2 \sum_k h_k \mathcal{F} \varphi(2t - k) \\
                            &= 2 (\sum_k h_k e^{-i \frac{\omega}{2}k}) \cdot \hat{\varphi}(\frac{\omega}{2}) \frac{1}{2} \\
                            &= H(\frac{\omega}{2}) \hat{\varphi}(\frac{\omega}{2})
    \end{align*} mit $H(\omega) \coloneqq \sum_k h_k e^{i\omega k}$
    
    Mit \autoref{MSA:v} folgt auch $\hat{\varphi}(0) = 1$, und damit $H(0) = \sum_k h_k = 1$. Man kann auch mit der Orthogonalität der Basen zeigen, dass $\{\varphi_{j,k}\}_k$ auch $H(\pi) = 0$ gelten muss. Damit wäre $H$ ein Tiefpaßfilter.

    \begin{lem}[Allgemeine Skallierungsgleichung]
        Es gilt \label{lem:Skal}
        $$\varphi_{j,k} = \sqrt{2} \sum_l h_l \varphi_{j+1, l+2k}$$
    \end{lem}

    \begin{proof} Per Definition gilt:
        \begin{align*}
            \varphi_{j, k} (t) &= 2^{j/2} \varphi(2^j t - k) \\
            &\stackrel{\autoref{lem:Skal}}{=} 2 \cdot 2^{j/2} \sum_l h_l \varphi(2^{j+1} t -2k -l) \\
            &= \sqrt{2} \sum_l h_l \varphi_{j+1, l + 2k}
        \end{align*}
    \end{proof}

    \section{Wavelets}

    Nun definieren wir uns die Komplementärräume zu einer MSA.

    \begin{defn}
        Für $\{V_j\}_j$ eine MSA ist $\psi$ ein Wavelet, falls $\psi(t-k)$ eine Orthonormalbasis, Riesz-Basis von $W_0$, mit $W_0$ der Komplementärraum von $V_0$ in $V_1$, also $V_1 = V_0 \oplus W_0$. Außerdem soll gelten 
        \begin{equation}
            \int_{-\infty}^\infty \psi(t) \, dt = 0 \label{eq:wav}
        \end{equation}
    \end{defn}

    Analog definieren wir uns
    $$\psi_{j, k}(t) \coloneqq 2^{j/2} \psi(2^j t - k)$$
    Auch hier gilt, dass $\{\psi_{j,k}\}_k$ eine Basis von $W_j$ ist.
    Damit gilt, dass $\psi \in V_1$, mit
    $$\psi (t) = 2 \sum_k g_k \varphi (2t - k)$$
    Analog zu oben folgern wir 
    $$\hat{\psi} (\omega) = G(\frac{\omega}{2}) \hat{\varphi}(\frac{\omega}{2})$$
    mit
    $$G(\omega) = \sum_k g_k e^{-ik\omega}$$
    Durch \autoref{eq:wav} folgt $\hat{\psi}(0) = 1 \: \hat{\varphi}(0) = 0$, und damit $G(0) = \sum_k g_k = 0$. Durch die Orthogonalität kann man auch hier zeigen, dass $G(1) = 1$. Damit ist $G$ ein Hochpaßfilter.

    \section{Die Fast Forward Wavelet Transformation}

    Ziel dieser Transformation ist es bei gegebener Detailschärfer (hier Koeffizienten aus einem Raum $V_j$) die Koeffizienten aus den gröberen Räumen zu berechnen.
    Hierbei nehmen wir als Koeffizienten die Projektion einer funktion $f \in \lR$ auf den Raum $V_j+1$, also mit $s_{j+1, k} = \langle f, \varphi_{j + 1, k}\rangle$:
    $$\sum_k s_{j+1, k} \varphi_{j+1, k} = \sum_k s_{j,k} \varphi_{j,k} + \sum_k \omega_{j, k} \psi_{j, k}$$
    Durch Skallarmultiplikation auf beiden Seiten bekommen wir für ein $l \in \mathbb{Z}$ fix.:
    \begin{align*}
        \sum_k s_{j+1, k} \langle \varphi_{j+1, k}, \varphi_{j, l} \rangle &= \sum_k s_{j,k} \langle \varphi_{j,k}, \varphi_{j, l} \rangle + \sum_k \omega_{j, k} \langle \psi_{j, k}, \varphi_{j, l} \rangle \\
        s_{j, l} &= \sum_k s_{j+1, k} \langle \varphi_{j+1, k}, \varphi_{j, l} \rangle 
    \end{align*}

    Durch $\varphi_{j,k} = \sqrt{2} \sum_l h_l \varphi_{j+1, l+2k}$ folgt:
    $$\langle \varphi_{j+1, l}, \varphi_{j, k} \rangle = \sqrt{2} \sum_m h_m \langle \varphi_{j+1, l}, \varphi_{j+1, m + 2k} \rangle = \sqrt{2} h_{l-2k}$$

    Somit gilt:
    $$s_{j, k} = \sqrt{2} \sum_l s_{j+1, l} h_{l - 2k}$$

    \begin{thebibliography}{99}
        \bibitem{texbook}
        Jöran Bergh, Fredrik Ekstedt, Martin Lindberg (1999) Wavelets mit Anwendungen in Signal- und Bildverarbeitung
    \end{thebibliography}
    \end{document}